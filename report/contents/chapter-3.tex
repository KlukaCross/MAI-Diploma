\section{Демонстрация программного продукта и анализ результатов}

\subsubsection{Обзор прототипа}
В рамках дипломной работы был разработан прототип системы в виде инструмента c интерфейсом командной строки. Рассмотрим его функциональные возможности.

Прототип обеспечивает перенос данных и их удаление из базы данных. Кроме того, данный инструмент позволяет осуществлять перенос и удаление схемы базы данных или выводить схему в формате диаграммы PlantUML \site{plantuml}.

В прототипе реализованы основные компоненты, указанные в \ref{Sequence DataTransferComponents}, включая элементы DataWalker и DataWriter. Процесс обхода данных основан на алгоритме, описанном в \ref{algorithm-with-rules}. Для записи данных из одной базы в другую применяется механизм Foreign Data Wrapper.

Кроме того, в прототипе предусмотрены две версии взаимодействия с базой данных и между компонентами GraphWalker и DataWriter: синхронная и асинхронная. В случае синхронного взаимодействия компонент GraphWalker ожидает завершения записи данных компонентом DataWriter, в то время как при асинхронном взаимодействии компоненты функционируют независимо друг от друга.

Также в прототипе поддерживается функциональность следующих конструкции для описания метаданных: \textit{GRAPH SOURCE}, \textit{NO ENTER}, \textit{NO EXIT}, \textit{LIMIT DISTANCE}.

\subsubsection{Примеры использования и производительность}

TBD: описать какой-нибудь пример с бд, описанной ранее.

TBD: наполнить базку множеством данных и потестить прототип на ней. Можно сравнить производительность синхронной и асинхронной версий, а также использовать pg\_dump

\subsubsection{Анализ результатов}

TBD: сделать анализ получившегося прототипа. Написать про жизнеспособность системы: можно ли пользоваться этой системой, либо проще использовать готовые инструменты, в том числе обычные SQL-запросы.
