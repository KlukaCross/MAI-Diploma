\introduction % Структурный элемент: ВВЕДЕНИЕ

В настоящее время рынок онлайн-образования демонстрирует значительный рост: объём российского рынка образовательных технологий в 2024 году составил 149 миллиардов рублей, увеличившись на 21\% по сравнению с 2023 годом \cite{online-edu}. Сегмент онлайн-образования, направленный на детей, составляет 29\% \cite{online-edu-child}. В связи с этим внимание следует уделить платформам, предназначенным для участников средней общеобразовательной школы.

Эти платформы требуют надёжной системы управления данными для обеспечения стабильной работы и поддержки большого количества пользователей. Одним из распространённых решений в области управления базами данных является PostgreSQL~\cite{postgresql} — реляционная система управления базами данных с открытым исходным кодом.

В процессе разработки платформы важно уделять особое внимание качеству и надёжности. Это делает тестирование незаменимой стадией разработки. Благодаря тестированию можно убедиться в стабильности и бесперебойной работе систем.

Использование реальных данных и проведение тестов в производственной среде позволяет максимально точно оценить работоспособность системы. Но такой подход сопряжён с рисками, такими как нарушения конфиденциальности данных и потенциальные сбои в работе системы.

Чтобы тестирование образовательных платформ было безопасным и в то же время позволяло бы максимально точно оценить работоспособность системы, необходимо создать тестовую среду, максимально приближенную к производственным условиям. Создание тестовой среды требует тщательной проработки и учёта всех аспектов, связанных с функционированием платформы. Это включает в себя настройку инфраструктуры, идентичной производственной, и воспроизведение всех взаимосвязей данных, которые присутствуют в реальных условиях эксплуатации.

Особую значимость в этом контексте приобретает работа с данными. Для достижения максимальной схожести с производственной средой необходимо использовать тестовые данные, которые точно отражают объёмы, структуры и взаимосвязи данных, присутствующие в реальной системе.

Создание тестового набора данных может быть реализовано с использованием различных методик. Один из подходов включает полный перенос данных из производственной базы данных с последующей анонимизацией конфиденциальной информации. Утилита pg\_dump~\cite{pg-dump} может быть использована для экспорта данных из PostgreSQL, обеспечивая перенос структуры и содержимого базы данных.

Для защиты пользовательской информации и анонимизации данных можно применить утилиту pg\_anonymizer~\cite{pg-anonymizer}, которая помогает скрыть чувствительные данные, заменяя их на фиктивные значения, что сохраняет конфиденциальность пользователей.

Основной недостаток данного метода заключается в невысокой скорости переноса всех данных, что может быть критичным при работе с крупными наборами данных и требовать значительных временных затрат для выполнения всей операции.

Частичный перенос данных может быть более подходящим для тестирования, так как зачастую не требуется полный объём данных для покрытия большинства сценариев. Но этот подход сопряжён с рядом вопросов и потенциальных сложностей, которые опишем ниже.

При полном переносе данных тестировщик получает доступ ко всем данным, что позволяет ему проводить тестовые сценарии. В случае частичного переноса данных требуется определить и описать необходимые для тестирования данные, избегая глубокого изучения структуры базы данных и её связей. Возникает вопрос, каким образом тестировщику описать данные, нужные для проведения тестов.

Следующим вопросом является сохранение целостности данных. В случае полного переноса реляционная целостность поддерживается автоматически, поскольку все взаимосвязанные данные переносятся полностью, обеспечивая корректность и согласованность системы. При частичном переносе необходимо уделить особое внимание поддержанию целостности данных. Недопустимо, чтобы после переноса в тестовой среде данных было недостаточно для выполнения тестовых сценариев. Следовательно, второй вопрос заключается в обеспечении целостности данных при минимизации объёма переносимых данных.

Перенос данных из производственной среды с использованием анонимизации представляет собой лишь один из способов получения тестовых данных. Альтернативный подход состоит в генерации данных на основе заданных характеристик. Этот процесс может выглядеть следующим образом. Пользователь задаёт параметры, которые могут сопровождаться набором или диапазоном значений. Затем программа генерирует тестовые данные, основываясь на фактических данных, либо на схеме данных производственной базы, а также на указанных пользователем характеристиках. Таким образом, полученные данные будут иметь структуру, аналогичную реальным данным, но не будут связаны с конкретными записями в производственной базе. Но для этого метода возникают вопросы описания пользователем необходимых данных и их корректной генерации.

Дипломная работа будет посвящена проектированию системы, которая должна поддерживать следующую функциональность:
\begin{itemize}
    \item перенос взаимосвязанных данных из одной базы данных в другую с применением методов анонимизации,
    \item генерацию тестовых данных.
\end{itemize}
В рамках работы будут рассмотрены алгоритмы переноса и генерации данных. Также будет разработан язык для описания данных и прототип системы.
