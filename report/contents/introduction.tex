\introduction % Структурный элемент: ВВЕДЕНИЕ

В настоящее время рынок онлайн-образования демонстрирует значительный рост. Всё большее число людей различных возрастных категорий предпочитают дистанционные форматы, которые обеспечивают гибкость и доступность обучения независимо от географического положения. Сегмент онлайн-образования, направленный на школьников, составляет значительную часть рынка.~\cite{online-edu-research} В связи с этим внимание следует уделить платформам, разработанным для школьников, их родителей и учителей.

Подобные системы позволяют объединить учебные материалы, задания, оценки и обратную связь в одном пространстве. Для школьников это обеспечивает обучение в интерактивной и увлекательной форме, для родителей — возможность контролировать достижения своих детей и участвовать в их образовательном процессе, а для учителей — эффективное управление учебным процессом и возможность персонализированного подхода в обучении.

Эти платформы требуют надёжной и эффективной системы управления данными для обеспечения стабильной работы и поддержки большого количества пользователей. Одним из распространённых решений в области управления базами данных является PostgreSQL~\cite{postgresql} — реляционная система управления базами данных с открытым исходным кодом.

В процессе разработки платформы важно не только работать над техническими аспектами, но и уделять особое внимание качеству и надежности. Это делает тестирование незаменимой стадией разработки. Благодаря тестированию можно убедиться в стабильности и бесперебойной работе систем.

Использование реальных данных и проведение тестов в производственной среде позволяет максимально точно оценить работоспособность системы. Но такой подход сопряжён с рисками, такими как нарушения конфиденциальности данных, потенциальные сбои в работе системы и возможное негативное влияние на реальных пользователей.

Чтобы тестирование образовательных платформ было безопасным и в то же время эффективным, необходимо создать тестовую среду, максимально приближенную к производственным условиям. Создание тестовой среды требует тщательной проработки и учета всех аспектов, связанных с функционированием платформы. Это включает в себя настройку инфраструктуры, идентичной производственной, и воспроизведение всех взаимосвязей данных, которые присутствуют в реальных условиях эксплуатации.

Особую значимость в этом контексте приобретает работа с данными. Для достижения максимальной схожести с производственной средой необходимо использовать тестовые данные, которые точно отражают объёмы, структуры и взаимосвязи данных, присутствующие в реальной системе.

Создание тестового набора данных может быть реализовано с использованием различных методик. Один из подходов включает полный перенос данных из производственной базы данных с последующей анонимизацией конфиденциальной информации. Утилита pg\_dump~\cite{pg-dump} может быть использована для экспорта данных из PostgreSQL, обеспечивая перенос структуры и содержимого базы данных.

Для защиты пользовательской информации и анонимизации данных можно применить утилиту pg\_anonymizer~\cite{pg-anonymizer}, которая помогает скрыть чувствительные данные, заменяя их на фиктивные значения, что сохраняет конфиденциальность пользователей.

Несмотря на эффективное использование таких средств, основной недостаток данного метода заключается в невысокой скорости переноса всех данных, что может быть критичным при работе с крупными наборами данных и требовать значительных временных затрат для выполнения всей операции.

Частичный перенос данных может быть более эффективным для тестирования, так как зачастую не требуется полный объём данных для покрытия большинства сценариев. Однако этот подход сопряжён с рядом вопросов и потенциальных сложностей.

При полном переносе данных тестировщик получает доступ ко всем данным, что позволяет ему проводить тестовые сценарии. В случае частичного переноса данных требуется определить и описать необходимые для тестирования данные, избегая глубокого изучения структуры базы данных и её связей. Возникает вопрос: каким образом тестировщику эффективно описать данные, нужные для проведения тестов?

Следующим вопросом является сохранение целостности данных. В случае полного переноса реляционная целостность поддерживается автоматически, поскольку все взаимосвязанные данные переносятся полностью, обеспечивая корректность и согласованность системы. При частичном переносе необходимо уделить особое внимание поддержанию целостности данных. Недопустимо, чтобы после переноса в тестовой среде данных было недостаточно для выполнения тестовых сценариев. Следовательно, второй вопрос заключается в следующем: как обеспечить целостность данных, минимизируя при этом объём переносимых данных?

Перенос данных из производственной среды с использованием анонимизации представляет собой лишь один из способов получения тестовых данных. Альтернативный подход состоит в генерации данных на основе заданных характеристик. Как может быть реализован этот процесс? Пользователь задаёт характеристики, такие как возраст, пол или средний балл по математике; каждый из параметров может сопровождаться набором или диапазоном значений. Программа затем генерирует тестовые данные, основываясь на фактических данных, либо на схеме данных производственной базы, а также на указанных пользователем характеристиках. Таким образом, полученные данные будут иметь структуру, аналогичную реальным данным, но не будут связаны с конкретными записями в производственной базе. Но для этого метода также возникают два ключевых вопроса: как пользователю описать необходимые данные и как их корректно сгенерировать.

Дипломная работа будет посвящена проектированию системы, обеспечивающей перенос взаимосвязанных данных из одной базы данных в другую на основе пользовательских запросов с применением методов анонимизации, а также способной генерировать тестовые данные. В рамках исследования будут рассмотрены алгоритмы переноса и генерации данных, а также разработка языка для описания данных, необходимых пользователю.
