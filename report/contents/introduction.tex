\introduction % Структурный элемент: ВВЕДЕНИЕ

В современном мире рынок онлайн-образования переживает бурное развитие. Всё больше людей всех возрастов обращаются к дистанционным форматам, которые предлагают гибкость и доступность обучения в любой точке земного шара. Сегмент онлайн-образования, ориентированного на школьников, занимает значительную долю рынка.~\cite{online-edu-research} В связи с этим особого внимания заслуживают платформы, предназначенные для школьников, их родителей и учителей.

Такие системы позволяют интегрировать учебные материалы, задания, оценки и обратную связь в едином пространстве. Для школьников это возможность учиться в интерактивной и увлекательной форме, для родителей — возможность отслеживать успехи своих детей и принимать участие в образовательном процессе, а для учителей — эффективное управление учебным процессом и персонализированное обучение.

Эти платформы требуют надежной и эффективной системы управления данными для обеспечения бесперебойного функционирования и поддержки большого количества пользователей. Одним из популярных решений в области управления базами данных является PostgreSQL~\cite{postgresql} — реляционная система управления базами данных с открытым исходным кодом. PostgreSQL завоевала доверие разработчиков благодаря своей надежности, богатому функционалу и способности обрабатывать большие объемы данных.

В процессе разработки платформы важно не только работать над техническими аспектами, но и уделять особое внимание качеству и надежности. Это делает тестирование незаменимой стадией разработки. Благодаря тестированию можно убедиться в стабильности и бесперебойной работе систем.

Использование реальных данных и проведение тестов в производственной среде позволяет максимально точно оценить работоспособность системы. Но такой подход сопряжён с рисками, такими как нарушения конфиденциальности данных, потенциальные сбои в работе системы и возможное негативное влияние на реальных пользователей.

Чтобы тестирование образовательных платформ было безопасным и в то же время эффективным, необходимо создать тестовую среду, максимально приближенную к производственным условиям. Создание тестовой среды требует тщательной проработки и учета всех аспектов, связанных с функционированием платформы. Это включает в себя настройку инфраструктуры, идентичной производственной, и воспроизведение всех взаимосвязей данных, которые присутствуют в реальных условиях эксплуатации.

Особенно важна тут роль работы с данными. Для достижения максимальной схожести с производственной средой необходимо иметь тестовые данные, которые бы точно отражали объемы, структуры и взаимосвязи данных в реальной системе.

Создание тестового набора данных может осуществляться посредством различных методик. Один из подходов подразумевает полный перенос данных из производственной базы данных и дальнейшую анонимизацию конфиденциальной информации. Данную операцию можно выполнить с помощью утилиты pg\_dump~\cite{pg-dump}, которая обеспечивает экспорт данных. Для защиты пользовательской информации процесс анонимизации можно выполнять при помощи утилиты pg\_anonymizer~\cite{pg-anonymizer}. Несмотря на эффективность выбора подобных средств, основным недостатком данного метода выступает невысокая скорость переноса данных.

Иногда базы данных в крупных программных продуктах могут содержать терабайты данных~\cite{pg-big-data-example}. В таких случаях процесс полного переноса данных способен занимать продолжительное время. Однако в ряде ситуаций столь длительное время, затрачиваемое на создание тестовых данных, является недопустимым.

Как правило, для покрытия большинства тестовых сценариев требуется не полный объём данных, а лишь их часть, что указывает на необходимость частичного переноса данных. Однако данный подход вызывает несколько вопросов.

В случае полного переноса тестировщику предоставляется доступ ко всему массиву данных, на основе которого он может проводить тестовые сценарии. При частичном переносе данных тестировщику необходимо каким-то образом определить и описать те данные, которые ему требуются, желательно не углубляясь в детали структуры базы данных и существующие в ней связи. Таким образом, первый вопрос заключается в следующем: каким образом тестировщику эффективно описать данные, необходимые для его работы?

Следующим важным вопросом является обеспечение целостности данных. При полном переносе реляционная целостность соблюдается автоматически, так как все взаимосвязанные данные переносятся в полном объёме, обеспечивая корректность и согласованность системы. Однако при частичном переносе необходимо серьезное внимание уделить поддержанию целостности данных. Недопустимо, чтобы после переноса в тестовую среду данных оказалось недостаточно для выполнения тестовых сценариев. Следовательно, второй вопрос заключается в следующем: как сохранить целостность данных, минимизируя при этом объём переносимых данных?

Перенос данных из производственной среды с использованием анонимизации является лишь одним из методов получения тестовых данных. Альтернативный метод заключается в генерации данных на основе заданных характеристик. Как может выглядеть такой сценарий? Пользователь определяет характеристики, такие как возраст, пол или средний балл по математике, при этом каждый параметр может сопровождаться набором или диапазоном значений. Далее программа генерирует тестовые данные, опираясь на сами данные, либо схему данных из производственной базы, а также на заданные пользователем характеристики. В результате полученные данные будут иметь структуру, аналогичную реальным данным, однако они не будут ассоциироваться с какими-либо конкретными записями из производственной базы. Тем не менее, для данного подхода также встают два ключевых вопроса: как пользователю описать требуемые данные и как их корректно сгенерировать.

Дипломная работа будет посвящена разработке системы, которая на основе пользовательских запросов обеспечит перенос взаимосвязанных данных из одной базы данных в другую с применением методов анонимизации, а также будет способна генерировать тестовые данные. В рамках исследования будут рассмотрены алгоритмы переноса и генерации данных, а также разработка языка для описания данных, необходимых пользователю.
