\section{Постановка задачи и теоретические предпосылки}

\subsection{Тестирование}

Для определения требований искомой системы необходимо формализовать процесс тестирования программного обеспечения.

\subsubsection{Процесс тестирования}

Тестирование программного обеспечения --  это процесс проверки и оценки работы программы или системы для выявления ошибок, дефектов и несоответствий. Основная цель тестирования -- удостовериться в том, что ПО работает корректно и удовлетворяет требованиям пользователей \cite{software-testing}.

Процесс тестирования можно разбить на несколько ключевых этапов \cite{rex-black}:

\begin{enumerate}
    \item планирование:
    \begin{itemize}
        \item разработка стратегии тестирования, включая выбор типов тестирования,
        \item создание тест-плана, где описываются ресурсы, сроки и критерии начала и завершения тестирования,
        \item проведение анализа рисков, связанных с тестированием, и определение действий для минимизации их воздействия,
    \end{itemize}

    \item подготовка:
    \begin{itemize}
        \item разработка тестовых случаев и сценариев на основе требований и спецификаций,
        \item подготовка тестовой среды, включая оборудование и программные средства, необходимые для проведения тестов,
        \item подготовка данных для тестирования и настройка автоматизированных тестов, если они предусмотрены,
    \end{itemize}

    \item проведение:
    \begin{itemize}
        \item непосредственное выполнение тестов согласно плану тестирования,
        \item фиксация и документирование результатов тестирования, выявление дефектов и несоответствий,
    \end{itemize}

    \item совершенствование:
    \begin{itemize}
        \item анализ результатов тестирования и составление отчёта по итогам выполненной работы,
        \item обмен опытом и полученными знаниями внутри команды и с другими заинтересованными сторонами,
        \item обновление и улучшение тестовых документов и процессов на основе полученных данных.
    \end{itemize}
\end{enumerate}

Наша система будет применяться на этапе подготовки: с её помощью можно будет создавать тестовые данные.

\subsubsection{Ручное и автоматизированное тестирование}

Тестирование можно разделить на ручное и автоматизированное \cite{testing-classes}.

Ручное тестирование предполагает процесс, при котором тестировщик выполняет тесты без помощи автоматизированных инструментов.
Оно позволяет глубже вникнуть в пользовательский опыт, и его легко применять к новым или часто изменяющимся функциональностям. Но ручное тестирование может занимать много времени, оно подвержено человеческим ошибкам и не всегда позволяет повторить результаты для их сравнения.

Автоматизированное тестирование предполагает использование программных инструментов для автоматизации выполнения тестовых сценариев. Это предполагает разработку скриптов, которые автоматически выполняют тесты и сверяют результаты с ожидаемыми.
Автоматизированное тестирование обеспечивает высокую скорость и стабильность результатов, что делает его эффективным для регрессионного тестирования и стабильных функциональностей. Но автоматизированное тестирование требует значительных ресурсов для разработки скриптов и не всегда подходит для тестирования пользовательских интерфейсов и динамически изменяющихся требований.

Оба подхода часто используются совместно для достижения наилучшего результата. Обычно ручное тестирование применяют на начальных стадиях или для исследовательского тестирования, тогда как автоматизированное тестирование использует для повторяющихся задач или сложных сценариев, которые требуют стабильного выполнения.

\subsection{Сценарии использования}

Перед определением требований к нашей системы рассмотрим возможные сценарии использования.

Определим роли:

\begin{itemize}
    \item тестировщик -- специалист, занимающийся проверкой качества ПО,
    \item разработчик -- специалист, занимающийся разработкой ПО,
    \item DevOps-инженер -- специалист, занимающийся поддержкой процессов CI/CD.
\end{itemize}

Будем называть \textbf{метаданными} входные данные, включающие в себя правила переноса, генерации или анонимизации данных.

\subsubsection{Перенос данных для проведения тестовых сценариев}

Роли:

\begin{itemize}
    \item тестировщик,
    \item разработчик.
\end{itemize}

Цель: создать тестовые данные на основе существующих реальных данных для их дальнейшего применения в тестовых сценариях.

Предпосылки:

\begin{itemize}
    \item разработчиком написаны метаданные, включающие в себя правила переноса и анонимизации данных,
    \item тестировщик осведомлен о метаданных, необходимых для проведения тестовых сценариев,
    \item тестировщик имеет уникальный идентификатор исходной базы данных,
    \item имеется пустая тестовая база данных, доступ к которой имеется у тестировщика.
\end{itemize}

Основной сценарий:

\begin{enumerate}
    \item тестировщик осуществляет изменения в метаданных,
    \item тестировщик инициирует запуск CLI, указывая следующие параметры:
    \begin{itemize}
        \item уникальный идентификатор исходной базы,
        \item уникальный идентификатор целевой базы или параметры подключения к ней,
        \item метаданные или уникальный идентификатор метаданных,
    \end{itemize}
    \item тестировщик ожидает завершения процесса переноса данных.
\end{enumerate}

Постусловия:

\begin{itemize}
    \item тестировщик применяет данные, сгенерированные на основе реальных данных, в тестовых сценариях.
\end{itemize}

Исключения:

\begin{itemize}
    \item тестировщик получает ошибку некорректных метаданных после пункта б.
\end{itemize}


\subsubsection{Диагностика ошибки}

Роли:

\begin{itemize}
    \item тестировщик.
\end{itemize}

Цель: провести диагностику ошибки, возникшей с определенными данными в продуктивной среде.

Предпосылки:

\begin{itemize}
    \item в продуктивной среде обнаружены проблемы с определенными данными, такими как данные пользователя,
    \item тестировщик владеет уникальным идентификатором данных, в которых выявлена ошибка в продуктивной среде,
    \item тестировщик имеет уникальный идентификатор исходной базы данных,
    \item имеется пустая тестовая база данных, доступ к которой имеется у тестировщика.
\end{itemize}

Основной сценарий:

\begin{enumerate}
    \item тестировщик описывает метаданные, включая данные, в которых произошла ошибка,
    \item тестировщик инициирует запуск CLI, указывая следующие параметры:
    \begin{itemize}
        \item уникальный идентификатор исходной базы,
        \item уникальный идентификатор целевой базы или параметры подключения к ней,
        \item метаданные,
    \end{itemize}
    \item тестировщик ожидает завершения процесса переноса данных.
\end{enumerate}

Альтернативный сценарий:

\begin{itemize}
    \item тестировщик может взять готовые метаданные, если они ему подходит, а не описывать свои.
\end{itemize}

Постусловия:

\begin{itemize}
    \item тестировщик осуществляет взаимодействие с перенесенными данными и проводит диагностику ошибки.
\end{itemize}

Исключения:

\begin{itemize}
    \item тестировщик получает ошибку некорректных метаданных после пункта б.
\end{itemize}


\subsubsection{Тестирование новой миграции}

Роли:

\begin{itemize}
    \item тестировщик,
    \item разработчик.
\end{itemize}

Цель: провести тестирование новой миграции.

Предпосылки:

\begin{itemize}
    \item тестировщик имеет уникальный идентификатор исходной базы данных,
    \item имеется пустая тестовая база данных, доступ к которой имеется у тестировщика,
    \item разработчик создал новую миграцию, нуждающуюся в тестировании,
    \item разработчик предоставил метаданные к миграции, содержащие правила анонимизации и переноса данных.
\end{itemize}

Основной сценарий:

\begin{enumerate}
    \item тестировщик инициирует запуск CLI, указывая следующие параметры:
    \begin{itemize}
        \item уникальный идентификатор исходной базы,
        \item уникальный идентификатор целевой базы или параметры подключения к ней,
        \item предоставленные метаданные,
    \end{itemize}
    \item тестировщик ожидает завершения процесса переноса данных.
\end{enumerate}

Постусловия:

\begin{itemize}
    \item тестировщик взаимодействует с перенесенными данными и осуществляет тестирование миграций.
\end{itemize}


\subsubsection{Перенос данных для автоматизированного тестирования}

Роли:

\begin{itemize}
    \item тестировщик,
    \item разработчик,
    \item DevOps-инженер.
\end{itemize}

Цель: обеспечеть возможность проведения автоматизированного тестирования изменений через сценарий CI/CD.

Предпосылки:

\begin{itemize}
    \item разработчиком написаны метаданные, включающие в себя правила переноса и анонимизации данных,
    \item тестировщиком реализованы автоматизированные тесты,
    \item DevOps-инженер разработал сценарий CI/CD, который, в случае появления нового запроса на слияния в системе контроля версий, инициирует развёртывание тестового окружения с пустой базой данных и осуществляет перенос необходимых данных с помощью нашей системы.
\end{itemize}

Основной сценарий:

\begin{enumerate}
    \item разработчик вносит изменения в исходный код и создает новый запрос на слияние в системе контроля версий,
    \item запускается процесс CI/CD, который выполняет следующие действия:
    \begin{itemize}
        \item развёртывание тестового окружения с пустой базой данных,
        \item перенос необходимых данных в тестовую базу с помощью нашей системы,
        \item автоматический запуск тестов для проверки корректности внесённых изменений.
    \end{itemize}
\end{enumerate}


\subsubsection{Генерация данных для проведения тестовых сценариев}

Роли:

\begin{itemize}
    \item тестировщик,
    \item разработчик.
\end{itemize}

Цель: создать тестовые данные на основе существующих реальных данных для их дальнейшего применения в тестовых сценариях.

Предпосылки:

\begin{itemize}
    \item разработчиком написаны метаданные, включающие в себя правила генерации данных,
    \item тестировщик осведомлен о метаданных, необходимых для проведения тестовых сценариев,
    \item тестировщик имеет уникальный идентификатор исходной базы данных,
    \item имеется пустая тестовая база данных, доступ к которой имеется у тестировщика.
\end{itemize}

Основной сценарий:

\begin{enumerate}
    \item тестировщик осуществляет изменения в метаданных,
    \item тестировщик инициирует запуск CLI, указывая следующие параметры:
    \begin{itemize}
        \item уникальный идентификатор исходной базы,
        \item уникальный идентификатор целевой базы или параметры подключения к ней,
        \item метаданные или уникальный идентификатор метаданных,
    \end{itemize}
    \item тестировщик ожидает завершения процесса генерации данных.
\end{enumerate}

Постусловия:

\begin{itemize}
    \item тестировщик применяет сгенерированные данные в тестовых сценариях.
\end{itemize}

Исключения:

\begin{itemize}
    \item тестировщик получает ошибку некорректных метаданных, после пункта б.
\end{itemize}


\subsection{Определение требований к системе}

Требования к системе можно разделить на две основные категории: функциональные и архитектурные свойства \cite{arch-requirements}. Функциональные требования описывают конкретный набор функций и возможностей, которые система должна обеспечивать для удовлетворения потребностей пользователей. Требования к свойствам архитектуры системы фокусируются на качественных характеристиках системы.

\subsubsection{Функциональные требования}

Из приведённых выше сценариев использования можно заключить, что система должна поддерживать следующую функциональность:

\begin{itemize}
    \item перенос и анонимизация данных. Система должна обеспечивать возможность переноса взаимосвязанных данных между базами данных и их анонимизацию. Правила выбора взаимосвязанных данных и правила анонимизации задаются метаданными;
    \item генерация данных. Система должна предоставлять функциональность генерации данных. Правила генерации задаются метаданными;
    \item работа с метаданными. Система должна осуществлять проверку корректности метаданных и обеспечивать функциональность для загрузки предварительно подготовленных метаданных или их интеграции на этапе ввода данных;
    \item гибкость в выборе базы данных. Необходимо внедрить функциональность, позволяющую пользователю выбрать базу данных, указав как уникальный идентификатор базы, так и строку подключения.
\end{itemize}

\subsubsection{Требования к свойствам архитектуры}

Существует множество архитектурных свойств \cite{fundamental-arch}.

Определим основные свойства, которым должна соответствовать система, в порядке приоритета:

\begin{itemize}
    \item безопасность. Архитектура должна обеспечивать защиту от несанкционированного доступа к подключаемым базам данных. Обеспечение безопасности системы является самым приоритетным свойством, потому что ответственность особенно высока при работе с персональными данными, так как это требует соблюдения строгих правовых норм;
    \item производительность. Архитектура должна обеспечивать высокую скорость переноса и генерации взаимосвязанных данных. Данное требование обусловлено недостаточной эффективностью существующих решений в области переноса данных, что было обосновано в вводной части дипломной работы;
    \item надёжность. Система должна функционировать без сбоев в течение продолжительных периодов времени. Это предполагает наличие мер по обеспечению отказоустойчивости и возможности восстановления после сбоев. Несмотря на потенциальное превосходство в производительности по сравнению с существующими аналогами, процессы переноса и генерации данных могут характеризоваться значительной временной продолжительностью. Прерывание данных процессов может привести к существенному замедлению процесса тестирования;
    \item асинхронная обработка запросов. Система должна поддерживать выполнение запросов в асинхронном режиме. Это означает, что клиентские приложения не должны блокироваться в ожидании завершения операции, но должны иметь возможность проверять статус выполнения запроса по мере необходимости.
\end{itemize}
