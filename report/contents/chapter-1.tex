\section{Постановка задачи и теоретические предпосылки}

\subsection{Тестирование}

Чтобы определить требования к нашей системе, для начала нужно разобраться с основными аспектами тестирования программного обеспечения.

\subsubsection{Процесс тестирования}

Тестирование — это процесс оценки системы или её компонентов с целью проверки их соответствия заданным требованиям. Основная цель тестирования заключается в выявлении дефектов, обеспечении качества и подтверждении работоспособности программного обеспечения.

Процесс тестирования можно разбить на несколько ключевых этапов~\cite{rex-black}:

\begin{enumerate}
    \item Планирование:
    \begin{itemize}
        \item разрабатывается стратегия тестирования, включая выбор типов тестирования;
        \item создается тест-план, где описываются ресурсы, сроки, и критерии начала и завершения тестирования;
        \item проводится анализ рисков, связанных с тестированием и определяются действия для минимизации их воздействия.
    \end{itemize}

    \item Подготовка:
    \begin{itemize}
        \item разработка тестовых случаев и сценариев на основе требований и спецификаций;
        \item подготовка тестовой среды, включая оборудование и программные средства, необходимые для проведения тестов;
        \item подготовка данных для тестирования и настройка автоматизированных тестов, если они предусмотрены.
    \end{itemize}

    \item Проведение:
    \begin{itemize}
        \item непосредственное выполнение тестов согласно плану тестирования;
        \item фиксация и документирование результатов тестирования, выявление дефектов и несоответствий.
    \end{itemize}

    \item Совершенствование:
    \begin{itemize}
        \item анализ результатов тестирования и составление отчёта по итогам выполненной работы;
        \item обмен опытом и полученными знаниями внутри команды и с другими заинтересованными сторонами;
        \item обновление и улучшение тестовых документов и процессов на основе полученных данных.
    \end{itemize}
\end{enumerate}

Наша система будет применяться на этапе подготовки: с её помощью можно будет создавать тестовые данные.

\subsection{Ручное и автоматизированное тестирование}

Тестирование можно классифицировать на ручное и автоматизированное.

Ручное тестирование предполагает процесс, при котором тестировщик выполняет тесты без помощи автоматизированных инструментов.
Оно позволяет глубже вникнуть в пользовательский опыт, и его легко применять к новым или часто изменяющимся функциональностям, однако оно может быть времяемким, подвержено человеческим ошибкам и не всегда позволяет повторить результаты для их сравнения.

Автоматизированное тестирование предполагает использование программных инструментов для автоматизации выполнения тестовых сценариев. Это предполагает разработку скриптов, которые автоматически выполняют тесты и сверяют результаты с ожидаемыми.
Автоматизированное тестирование обеспечивает высокую скорость и стабильность результатов, что делает его эффективным для регрессионного тестирования и стабильных функциональностей, однако требует значительных ресурсов для разработки скриптов и не всегда подходит для тестирования пользовательских интерфейсов и динамически изменяющихся требований.

Оба подхода часто используются совместно для достижения наилучшего результата. Обычно ручное тестирование применяют на начальных стадиях или для исследовательского тестирования, тогда как автоматизированное тестирование использует для повторяющихся задач или сложных сценариев, которые требуют стабильного выполнения.

\subsection{Сценарии использования}

\subsection{Определение требований к системе}
