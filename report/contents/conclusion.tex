\conclusion

В ходе дипломной работы была спроектирована система переноса и генерации взаимосвязанных данных. Были детально описаны алгоритмы выбора и генерации взаимосвязанных данных, разработан специфальный язык для описания метаданных, а также создан прототип системы. Дополнительно проведено тестирование производительности прототипа и осуществлён анализ жизнеспособности системы.

Несмотря на то, что проектирование и разработка проводились при тестировании образовательной платформы и основное тестирование прототипа осуществлялось с использованием одной базы данных, предлагаемая система обладает универсальностью и может быть интегрирована в другие платформы и применена в различных областях. Важным аспектом архитектуры системы является то, что функции, характерные для работы с PostgreSQL, инкапсулированы в одном компоненте. Это способствует легкой адаптации системы к другим реляционным СУБД.

В рамки работы не вошли некоторые детали. В частности, были опущены схемы базы данных, применяемых в системе, не были рассмотрены контейнеры Data Generator и Operation Manager на уровне компонент. Кроме того, в прототипе реализован ещё один алгоритм переноса данных, который в работе рассмотрен не был. Недостаточное внимание было уделено вопросам генерации данных и обеспечения безопасности системы.

В процессе разработки и тестирования прототипа возникали не только технические трудности, но и новые идеи, направленные на улучшение системы. Одной из таких идей стала разработка правил метаграфа, рассмотренных в данной работе. Но остались нерассмотренные идеи. Например, возникают сложности в понимании структуры базы данных и алгоритма обхода, что делает трудоёмким написание эффективных метаданных. Возможным решением этой проблемы может стать создание инструмента для анализа структуры базы данных, аналогичного Jailer, который бы поддерживал описанный алгоритм обхода и правила метаграфа. Такой инструмент обеспечил бы пользователю возможность визуального анализа структуры базы данных, а также позволил бы запускать алгоритм и легко описывать правила метаграфа.
