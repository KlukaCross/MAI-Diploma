\conclusion

В ходе дипломной работы была спроектирована система переноса и генерации взаимосвязанных данных. Были детально описаны алгоритмы выбора и генерации взаимосвязанных данных, разработан специфальный язык для описания метаданных, а также создан прототип системы. Дополнительно проведено тестирование производительности прототипа и осуществлён анализ жизнеспособности системы.

Несмотря на то, что проектирование и разработка проводились при тестировании образовательной платформы и основное тестирование прототипа осуществлялось с использованием одной базы данных, предлагаемая система обладает универсальностью и может быть интегрирована в другие платформы и применена в различных областях. Важным аспектом архитектуры системы является то, что функции, характерные для работы с PostgreSQL, инкапсулированы в одном компоненте. Это способствует легкой адаптации системы к другим реляционным СУБД.

В процессе разработки и тестирования прототипа были выявлены места, требующих улучшения:
\begin{itemize}
    \item первая проблема заключается в возникновении сложностей в понимании структуры базы данных и алгоритма обхода, что делает трудоёмким написание метаданных, которые бы удовлетворяли потребности пользователя. Возможным решением этой проблемы может стать создание инструмента для анализа структуры базы данных, аналогичного Jailer, который бы поддерживал описанный алгоритм обхода и правила метаграфа. Предполагаемый инструмент обеспечит пользователю возможность эффективной визуализации структуры базы данных и упростит процесс описания правил метаграфа;
    \item вторая проблема заключается в низкой производительности программы, особенно при больших объёмах данных. Есть гипотеза, что низкая производительность связана с частыми сетевыми запросами в базы данных. В качестве потенциального решения данной проблемы предлагается модификация взаимодействия с базой данных путём замены множественных мелких запросов на более редкие, но масштабные обращения к данным.
\end{itemize}
